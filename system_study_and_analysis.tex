% chktex-file 8
\chapter{SYSTEM STUDY AND ANALYSIS}
\section{Existing System}  
The existing system in the realm of chatbots primarily revolves around Large Language Models (LLMs). These models have revolutionized the field of natural language processing (NLP), enabling machines to understand and generate human-like text at an unprecedented scale.

\subsection{Large Language Models (LLMs)}
LLMs are a type of artificial intelligence model designed to generate and understand human-like text by analyzing vast amounts of data. These foundational models are based on deep learning techniques and typically involve neural networks with many layers and a large number of parameters, allowing them to capture complex patterns in the data they are trained on.

The primary goal of an LLM is to understand the structure, syntax, semantics, and context of natural language, so it can generate coherent and contextually appropriate responses or complete given text inputs with relevant information. These models are trained on diverse sources of text data, including books, articles, websites, and other textual content, which enables them to generate responses to a wide range of topics.

Two popular LLMs are BERT (Bidirectional Encoder Representations from Transformers) developed by Google, and GPT-3 and GPT-4 developed by OpenAI\@. BERT marked a departure from the prevalent natural language processing (NLP) approach that relied on recurrent neural networks (RNNs). In contrast, BERT is trained bidirectionally, allowing it to gain a more comprehensive understanding of language context and flow compared to its unidirectional predecessors.

GPT-3, or Generative Pre-trained Transformer 3, has garnered significant attention for its remarkable capabilities in natural language understanding and generation1. It became publicly used when developed into GPT-3.5 for the creation of the conversational AI tool ChatGPT\@. The largest language model is now OpenAI's GPT-4, released in March 20231.

\subsection{ChatGPT}
ChatGPT is a chatbot developed by OpenAI and launched on November 30, 202234. Based on a large language model, it enables users to refine and steer a conversation towards a desired length, format, style, level of detail, and language4. It works by predicting the next word in a given text, based on the patterns it has learned from a massive amount of data during its training process5.

However, the effectiveness of these LLMs is hindered by concerns surrounding bias, inaccuracy, and toxicity, which limit their broader adoption and raise ethical concerns1. Despite these challenges, the existing system of LLMs has made significant strides in the field of AI and continues to evolve.
\subsection{Drawbacks of Existing System}
The existing system of Large Language Models (LLMs) and chatbots like ChatGPT, while impressive, have several limitations:
\begin{enumerate}
  \item \textbf{Limited Understanding}: LLMs may have a limited understanding of the context and meaning of the language they process. They lack the ability to understand nuances, cultural references, and context-dependent meanings. This can lead to inaccurate or inappropriate responses to certain inputs. For example, they might not understand the difference between a genuine request for information and a rhetorical question, leading to responses that may seem out of place or irrelevant.
  \item \textbf{Incorrect Answers}: LLMs can make mistakes, including grammatical, mathematical, factual, and reasoning errors. They sometimes struggle to acknowledge their lack of knowledge and instead fabricate plausible-sounding answers. This can be problematic in situations where accurate information is crucial, such as in academic or professional settings. For instance, an LLM might generate an incorrect mathematical solution or provide outdated or incorrect factual information.
  \item \textbf{Biased Answers}: LLMs can reflect the biases present in their training data. This can lead to the perpetuation of stereotypes and biases. For example, if the training data contains gender biases, the LLM might generate responses that reinforce these biases. This is a significant issue as it can contribute to the spread of harmful stereotypes and discrimination.
  \item \textbf{Lack of Human Insight:}: LLMs lack the ability for genuine expression. They cannot produce content that touches people emotionally on the same level as a human can, as they have no actual thoughts or feelings. This means that while they can generate text that might seem insightful or empathetic, it's important to remember that these are simply outputs based on patterns in the data they were trained on, not genuine expressions of emotion or understanding.
  \item \textbf{Lack of Transparency}: LLMs like GPT-3 are highly complex and difficult to interpret. This makes it challenging to understand how they arrive at certain outputs. This lack of transparency can make it difficult to troubleshoot or improve the models, and it can also raise ethical concerns about accountability and fairness.
  \item \textbf{Ethical Concerns}: The use of LLMs raises ethical concerns, particularly around the generation of misinformation and the potential misuse of the technology. For example, they could be used to generate fake news or deceptive content, which could have serious societal implications.
  \item \textbf{Environmental Impact}: Training and running LLMs require significant computational resources, which can have a substantial environmental impact. The energy required to train these models and generate responses can contribute to carbon emissions, which is a significant concern given the ongoing climate crisis.

\end{enumerate}

\section{Proposed System}
\textbf{Retrieval-Augmented Generation (RAG)}\cite{lewis2021retrievalaugmented} is a method that combines the strengths of both retrieval-based and generative models for natural language understanding and generation.

\subsection{Working of RAG}
The working of RAG can be divided into several steps:
\begin{enumerate}
  \item \textbf{Data Loading}: The first step involves loading the dataset which is used for training the model. This dataset typically consists of a large number of documents or dialogues which provide the model with a diverse range of language patterns and structures. The data is usually preprocessed and tokenized before being fed into the model
  \item \textbf{Document Retrieval}: Once the data is loaded, the next step is document retrieval. When a new input (like a user query) is received, the model retrieves a set of relevant documents from the dataset. This is done using a dense retrieval method, which uses semantic similarity to find the most relevant documents. The semantic similarity is calculated using vector representations of the documents and the query. These vector representations are usually obtained using methods like Word2Vec or BERT, which convert the text into high-dimensional vectors. The similarity between the vectors is then calculated using methods like cosine similarity.
  \item \textbf{Question-Answering}: After the relevant documents are retrieved, a question-answering model is used. This model takes the user query and the retrieved documents as input, and generates a set of candidate responses. The question-answering model is usually a transformer-based model, which uses attention mechanisms to understand the context of the query and the documents, and generate relevant responses.
  \item \textbf{Response Generation}: Finally, a generative model is used to generate a response. This model takes the candidate responses from the previous step and generates a final response. This response is generated in a way that it is contextually relevant to the user query and is fluent and coherent in terms of language. The generative model is also usually a transformer-based model, which uses the context of the conversation and the candidate responses to generate a response.
\end{enumerate}
The key aspect of RAG is that it combines the strengths of retrieval-based and generative models. The retrieval component allows the model to leverage the information in the dataset, while the generative component allows the model to generate fluent and contextually relevant responses.
\subsection{Advantages of Proposed System}
Retrieval-Augmented Generation (RAG) offers several advantages that help overcome the limitations of traditional large language models (LLMs) like GPT-3. Here are some key advantages:
\begin{enumerate}
  \item \textbf{Contextual Understanding}: RAG models have a better understanding of the context as they retrieve relevant documents from a dataset before generating a response. This allows them to provide more accurate and contextually relevant responses compared to traditional LLMs, which generate responses based solely on the input they receive.
  \item \textbf{Information Retrieval}: One of the limitations of traditional LLMs is their inability to retrieve and leverage external information during the conversation. RAG models overcome this by retrieving relevant documents from a dataset, allowing them to provide responses that are not only based on the model's training data but also on the specific information present in the retrieved documents.
  \item \textbf{Dynamic Knowledge Update}: Traditional LLMs are trained on a static dataset, and their knowledge is fixed at the time of training. In contrast, RAG models can dynamically update their knowledge by retrieving information from a constantly updated dataset. This allows them to provide up-to-date information in their responses.
  \item \textbf{Scalability}: RAG models are scalable as they separate the retrieval and generation processes. This allows them to handle larger datasets and generate more diverse responses. In contrast, traditional LLMs may struggle with scalability as they need to process the entire dataset for every input.
  \item \textbf{Efficiency}: By separating the retrieval and generation processes, RAG models can efficiently handle complex tasks. They first narrow down the relevant information through retrieval and then focus on generating a response based on the retrieved information. This makes them more efficient compared to traditional LLMs, which need to consider the entire dataset while generating a response.
\end{enumerate}
In conclusion, RAG offers a robust solution for building chatbots by effectively combining the strengths of retrieval-based and generative models. It leverages the power of recent advancements in AI and LLMs to provide accurate, contextually relevant, and up-to-date responses.
\section{Preparation of Requirements}
The development of the chatbot for the college website required careful planning and preparation. The requirements were gathered and analyzed to ensure that the chatbot would meet the needs of the users and align with the goals of the project. The following are the key requirements that were identified:
\subsection{Functional Requirements}
\begin{enumerate}
  \item \textbf{User Input Processing}: The chatbot should be able to process user inputs in the form of text queries, understand the intent behind the queries, and generate appropriate responses.
  \item \textbf{Information Retrieval}: The chatbot should be able to retrieve relevant information from the college website to provide accurate and up-to-date responses to user queries.
  \item \textbf{Response Generation}: The chatbot should be able to generate fluent and contextually relevant responses based on the user queries and the retrieved information.
  \item \textbf{User Interaction}: The chatbot should be able to engage in natural and meaningful conversations with users, providing a seamless and user-friendly experience.
  \item \textbf{Multi-turn Conversations}: The chatbot should be able to handle multi-turn conversations, where the user can ask follow-up questions or continue the conversation on a specific topic.
  \item \textbf{Error Handling}: The chatbot should be able to handle errors and provide appropriate responses when it encounters queries that it cannot understand or process.
  \item \textbf{User and Chat History Management}: The chatbot should be able to handle user data and chat history, storing them securely in the database for future reference.
\end{enumerate}
\subsection{Non-Functional Requirements}
\begin{enumerate}
  \item \textbf{Accuracy}: The chatbot should provide accurate and reliable information to users, ensuring that the responses are based on the most up-to-date information available on the college website.
  \item \textbf{Performance}: The chatbot should be able to process user queries and generate responses in a timely manner, providing a responsive and efficient user experience.
  \item \textbf{Security}: The chatbot should adhere to data privacy and security standards, ensuring that user data is protected and handled in a secure manner.
  \item \textbf{User Experience}: The chatbot should provide a seamless and intuitive user experience, with a user-friendly interface and natural language interactions.
\end{enumerate}
\subsection{Technical Requirements}
\begin{enumerate}
  \item \textbf{Frontend}: The chatbot's user interface should be built using Vue.js to ensure a responsive and interactive user experience.
  \item \textbf{Backend}: The backend should be built using FastAPI to handle requests and responses efficiently. The backend should also incorporate the Llama index for handling the LLM and RAG\@.
  \item \textbf{Database}: PostgreSQL should be used as the database to store and manage data due to its robustness and reliability. The database should have tables for users and chat history.
\end{enumerate}
\section{Feasibility Study}
Before the commencement of the project, a feasibility study was conducted to assess the viability of the chatbot project for the college website. The study focused on the following areas:
\subsection{Technical Feasibility}
The project is technically feasible as it leverages Vue.js for frontend development, which is known for its simplicity and flexibility. FastAPI, a modern, fast (high-performance), web framework for building APIs is used for backend development. It has a significant advantage in terms of speed and ease of use. The Llama index is used for handling the LLM and RAG, and PostgreSQL, a powerful, open-source object-relational database system, is used for data storage. Additionally, OpenAI's GPT-3.5 and their embedding model are used for language understanding and generation, which are known for their high performance and accuracy. These technologies are well-documented and widely supported, which makes the development process efficient and manageable.
\subsection{Economic Feasibility}
From an economic perspective, the project is feasible as the technologies used (Vue.js, FastAPI, Llama index, PostgreSQL) are open-source and free to use. While OpenAI's models are not open-source, they come at a reasonable cost.
\subsection{Operational Feasibility}
The proposed chatbot is operationally feasible as it is designed to interact with users in a natural, conversational manner, and assist them in navigating the college website and finding the information they need. It can handle user data and chat history, storing them securely in the database for future reference.
\subsection{Legal and Ethical Feasibility}
The project complies with all relevant data protection and privacy laws. User data is handled securely, and privacy is maintained.
\subsection{Schedule Feasibility}
Given the scope of the project and the technologies used, the development of the chatbot is achievable within the proposed timeline.
\section{Data Flow Diagram}
To be filled 

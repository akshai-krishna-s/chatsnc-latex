
\chapter{INTRODUCTION}
In the rapidly evolving digital landscape, the demand for intelligent and context-aware systems is on the rise. Chatbots, in particular, have emerged as a powerful tool for providing information and assistance across various domains. However, the effectiveness of a chatbot is often limited by its ability to access and utilize the most relevant and up-to-date information. Recognizing this limitation, we introduce “ChatSNC”, a project that leverages the power of Retrieval Augmented Generation (RAG) to significantly enhance the capabilities of a chatbot.

RAG, a technique that combines information retrieval with text generation, allows AI models to retrieve relevant information from a knowledge source and incorporate it into the generated text. By using RAG, ChatSNC can query the college website and generate responses that are grounded in the official and up-to-date information of the college. This approach offers a significant advantage over traditional chatbots that rely solely on pre-defined scripts or large language models (LLMs), enabling ChatSNC to provide more accurate, informative, and engaging responses.

\section{About the project}  

ChatSNC is an innovative chatbot designed to interact with users on the college website. The frontend of the project is built using Vue.js, a progressive JavaScript framework known for its adaptability and versatility. The backend is developed using FastAPI, a modern, high-performance web framework for building APIs with Python based on standard Python type hints.

The core of ChatSNC lies in its use of the Retrieval Augmented Generation (RAG) and a large language model (LLM), powered by OpenAI's GPT-3.5. The RAG model allows the chatbot to retrieve relevant information from a knowledge source and incorporate it into the generated text, thereby providing responses that are grounded in the official and up-to-date information of the college.

The project also utilizes the Llama Index for handling the LLM and RAG, and PostgreSQL for data management. The database comprises tables for managing users and storing chat history, ensuring all interactions are securely logged and easily retrievable.

In essence, ChatSNC aims to revolutionize the way colleges communicate, making it more efficient, interactive, and user-friendly. By leveraging cutting-edge technologies and models, ChatSNC stands as a testament to the potential of AI in transforming digital communication in the educational sector.
